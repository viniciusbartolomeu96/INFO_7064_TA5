\documentclass[sigconf,nonacm]{acmart}

\usepackage[utf8]{inputenc}
\usepackage[T1]{fontenc}
\usepackage{graphicx}
\usepackage{booktabs}
\usepackage{amsmath}
\usepackage{microtype}
\usepackage{subcaption}

\bibliographystyle{ACM-Reference-Format}

\begin{document}

\title{TA05 - Exprerimentos com o MLP from scratch}

\author{Luan de Oliveira Magalhães}
\email{luan3642@hotmail.com}
\affiliation{%
  \institution{Departamento de Informática\\Universidade Federal do Paraná}
  \city{Curitiba}
  \state{Paraná}
  \country{Brasil}}

\author{Raul José Silvério da Silva}
\email{raul.silverio@ufpr.br}
\orcid{0009-0006-5091-1584}
\affiliation{%
  \institution{Departamento de Informática\\Universidade Federal do Paraná}
  \city{Curitiba}
  \state{Paraná}
  \country{Brasil}}

\author{Vinícius Lázaro Bartolomeu}
\email{vinicius.bartolomeu@ufpr.br}
\affiliation{%
  \institution{Departamento de Informática\\Universidade Federal do Paraná}
  \city{Curitiba}
  \state{Paraná}
  \country{Brasil}}

\begin{abstract}
Este relatório apresenta a resolução da quinta tarefa do projeto da disciplina de Visão Computacional e Percepção, cujo objetivo é estudar e analisar a implementação de uma Rede Neural do tipo Perceptron Multicamadas (MLP). Os experimentos foram conduzidos com o conjunto de dados load iris, fornecido pela biblioteca sklearn.datasets. A atividade contempla a avaliação do desempenho da MLP em tarefas de classificação, bem como a análise de sua convergência, os padrões de erro ao longo do treinamento e os efeitos do overfitting.
\end{abstract}

\keywords{Visão Computacional, classificação, iris}

\maketitle


%--------------------------------------------------------------------
\section{Transformação de Perspectiva}


%--------------------------------------------------------------------

\section{Método}


%--------------------------------------------------------------------

\section{Resultados}


%--------------------------------------------------------------------


\section{Conclusão}


\bibliography{references}

\end{document}
